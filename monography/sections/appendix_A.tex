CGSuite, which stands for Combinatorial Game Suite, is a software suite used to study and research combinatorial games. The tool implements all the common methods in the field of study, which includes, but is not limited to, all the ones described in this text. CGSuite is widely used by people interested in Combinatorial Game Theory.

The package also includes a graphical interface and a interpreter, for its own scripting language, together with the game analysis functionality. CGSuite is open source software and the development is made on github and releases are published on sourceforge. The core functionality is implemented in Java and recent updates are also developed in scala. The maintainer of the project is Aaron Nathan Siegel, the autor of \textit{Combinatorial Game Theory} \cite{CGT}.

The typical use of the software is through the GUI and consists of manipulating small games, in the sense that results can be calculated in reasonable time, drawing them and their thermographs. CGSuite also allows users to extend the provided games and implement new games while still making use of the tools available. Extending the package is usually done using the same scripting language. 

The software is powerful enough to handle large cases sometimes, but in the cases such as evaluating a truly large instance or several large instances a personalized solution may be needed. Unfortunately there are no other complete implementations yet. There is ongoing development of another Surreal Numbers library, written in julia, but nothing that would replace a personalized solution.







