\renewcommand{\textflush}{flushepinormal}
\setlength\epigraphwidth{.8\textwidth}
\epigraph{``I learned very quickly that playing games and working on mathematics were closely intertwined activities for him, if not actually the same activity. His attitude resonated with and affirmed my own thoughts about math as play, though he took this attitude far beyond what I ever expected from a Princeton math professor, and I loved it."}{Manjul Bhargava \footnotemark}

\footnotetext{Fields medalist commenting on John H. Conway's passing}

It is no surprise that avid players of games that resemble a logical or mathematical puzzle, like checkers, develop an intuition that allows them to calculate faster, by asserting bad moves fast, recognizing winning and losing patterns ahead of time or many other strategies. Many times, it is not easy to understand or explain how does it work, while others, after seeing a move being played, it is easily possible to explain it by the principles that rule the game. A most essential, and sometime very hard, component of playing well any of these mathematical games is being able to know if you are ahead or behind in a given position.

While evaluating whether a position is winning or losing is already a hard task, with, for example, modern chess engines not agreeing in some positions, it is possible take this problem to higher standards. Imagine you are playing a variant of the game of chess. In this variant, each player is given a set of board positions, and each should choose one board to play as white. During this game, play will take place in each board in parallel, and, whoever checkmates the opponent faster, wins the game. If one wants to be a great player of such game, the knowledge of asserting if a position is winning or loosing in a regular chess game is not enough, nor is the ability to play regular chess perfectly.

The most important ability for this game and for the games that will follow in this paper is to score each position. The ability to calculate the precise advantage a player has over the other is the object of interest of the field called Combinatorial Game Theory, inaugurated in \todo{1970} by the paper \todo{ONAG}. The author of that paper, Jonh H. Conway, as found in the epigraph that starts this text, was, as many, an avid player of such games, however, unlike all others, was one of the brightest minds in the history of mathematics.

Dr. Conway realized that some games, to be defined in the next chapter, behave like numbers in every aspect. While, initially, that seems very useful to calculate the advantages, and win in the presented variation of chess, this realization means much more than this tool. It is true that some way of modeling the problem of assigning values to position was the initial challenge and this would be a great step towards it. However, what came out from Conway's work was a \todo{match} between positions and numbers, not a measure of the position.




---------------------------------------------------------\\
History of field of study
- Quem inaugurou
-  Conway. Onag (paper que iniciou a area)
- Conway. WW (winning ways – referencia canonica). Morreu esse ano
- Combinatorial game theory
- Aron S. (abordagem diferente – coisas q n  tem no WW; especialmente em relacao a classificacao dos jogos)
- Principais nomes da area
- Conway + 2 
- dialogar com titulo: hot (nao eh conceito inicial da area) / games (nao eh definicao obvia)
- hot: temperatura/topico avancado: Conway.
- Traduzir intuicao em definicoes matematicas
- Definir o que eh games: games = mathematical play. Especificar.
- Numbers

\textcolor{red}{(The Introduction chapter should contain background information as appropriate, plus definitions of all special and general terms. Your topic should be: clearly stated and defined; have a clear overall purpose; and have clear, relevant and coherent aims and objectives. It is also informative to give a brief description of the contents of the remaining chapters of the thesis. This alerts the reader and prepares them for the rest of the thesis.)}


