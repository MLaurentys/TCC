Combinatorial game theory is a new field in mathematics that was initially found in the blend of recreational mathematics, combinatorics and number theory. While the initial focus is finding strategies to play better and win games, the developments it prompted in other areas of mathematics have their own place, independent of the initial purpose. The Surreal Numbers form a class with many interesting properties and found much enthusiasm for that reason, not because it was necessary to analyze games.

However, the surreal numbers inevitably carry the simplicity inherited from the mathematical plays it was created/discovered for. It is commonplace to learn how to breach the gap between rational and real numbers only in superior education, partially because any of the common constructions involves operations and procedures not used in school. Surreal Numbers give for free the construction of real numbers from its few and simple rules. Not only the reals, but a definition of positivity and negativity that dismisses comparison, a common method for the creation of all the numbers, new numbers and, at last, a direct way of visualizing each number with RB-Hackenbush games. Numbers are the main building block the text used to analyze games.

The non-numbers, a name used in this text for games with temperatures, are the remaining games and required more elements and concepts to be understood. The thermograph, or the form commonly used to visualize temperature, options, stops and boiling points, is extensively analyzed and the text presents one of the most recent related results.

The reader that understands all the featured topics and is able to replicate calculations and results to games other than the mentioned might carry the false belief that understands all of combinatorial game theory. This text, however, only describes the class of partisan games leaving many variations even unmentioned. Not only the focus is specific, but also not all perspectives necessary to tackle the problem of finding the best move are visited.

These information serves to show that there is much more to the field, not that little progress understanding it was made. In fact, it is fair to state that the classes of games discussed in the text are the more complicated ones and many of the definitions apply to other classes.

Both in the results discussed and the ones omitted there is a common trend in this field. The definitions are always simple, but powerful enough to allow many questions, and, although there are many open questions, the proofs are and solutions are simple as well. The procedure to build all dyadic rationals in domineering boards, for example, is based on a theorem provable in two lines of text and applying it is the simplest of ways. Although its explanation is simple, it remained open for years and the resulting pattern is not something that would be found before the proof.

Finding the upper bound to temperature is another case where the question remained open for long, but the strategy used to answer is simplicity: it consists only of listing and manipulating properties of the thermographs.That is not to say that the bounding of temperature is completely solved but the progress made used simple steps.

From this perspective combinatorial game theory is a beautiful field in mathematics. The definitions it is based upon are extremely few in number and very simple in form, but they allow proving more complex problems in a uncomplicated manner. However, in this text, one of the unvisited places in the field is computational complexity. In practice, calculating values is not an easy problem.

The other featured aspect skipped is the variety of forms combinatorial games might take, because the text only deals with partisan games played on normal conditions. The remaining contains a brief description of these two skipped problems and points toward texts on th respective subjects. This chapter, that finalizes the text, also addresses a few of the next steps studying or researching the field and ends with a personal reflection on the learning, implementing and writing experiences involved in this work.


\subsection*{Undeveloped Pieces}

Description + reference for complexity of combinatorial games

Description + reference for Impartial games and octals

\subsection*{Post-Match}

The choice of Combinatorial Game Theory as subject was an extremely good call. It is never enough to reiterate how  

%There are a few reasons this thesis ended up being about Combinatorial Game Theory and not some other subject. The first step in this directions was a great interest in the games of Chess, Shogi and Go. About three years ago, when first playing seriously board games, there was a desire to play better and improve.
%
%Even though playing was enjoyable, the initial ambition faded for a couple years, as there were more interesting subjects and classes to study. 
%













