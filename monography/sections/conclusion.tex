Combinatorial game theory is a new field in mathematics that was initially found in the blend of recreational mathematics, combinatorics and number theory. While the initial focus is finding strategies to play better and win games, the developments it prompted in other areas of mathematics have their own place, independent of the initial purpose. The Surreal Numbers form a class with many interesting properties and found much enthusiasm for that reason, not because it was necessary to analyze games.

However, the surreal numbers inevitably carry the simplicity inherited from the mathematical plays it was created/discovered for. It is commonplace to learn how to breach the gap between rational and real numbers only in superior education, partially because any of the common constructions involves operations and procedures not used in school. Surreal Numbers give for free the construction of real numbers from its few and simple rules. Not only the reals, but a definition of positivity and negativity that dismisses comparison, a common method for the creation of all the numbers, new numbers and, at last, a direct way of visualizing each number with RB-Hackenbush games. Numbers are the main building block the text used to analyze games.

The non-numbers, a name used in this text for games with temperatures, are the remaining games and required more elements and concepts to be understood. The thermograph, or the form commonly used to visualize temperature, options, stops and boiling points, is extensively analyzed and the text presents one of the most recent related results.

The reader that understands all the featured topics and is able to replicate calculations and results to games other than the mentioned might carry the false belief that understands all of combinatorial game theory. This text, however, only describes the class of partisan games leaving many variations even unmentioned. Not only the focus is specific, but also not all perspectives necessary to tackle the problem of finding the best move are visited.

These information serves to show that there is much more to the field, not that little progress understanding it was made. In fact, it is fair to state that the classes of games discussed in the text are the more complicated ones and many of the definitions apply to other classes.

Both in the results discussed and the ones omitted there is a common trend in this field. The definitions are always simple, but powerful enough to allow many questions, and, although there are many open questions, the proofs are and solutions are simple as well. The procedure to build all dyadic rationals in domineering boards, for example, is based on a theorem provable in two lines of text and applying it is the simplest of ways. Although its explanation is simple, it remained open for years and the resulting pattern is not something that would be found before the proof.

Finding the upper bound to temperature is another case where the question remained open for long, but the strategy used to answer is simplicity: it consists only of listing and manipulating properties of the thermographs.That is not to say that the bounding of temperature is completely solved but the progress made used simple steps.

From this perspective combinatorial game theory is a beautiful field in mathematics. The definitions it is based upon are extremely few in number and very simple in form, but they allow proving more complex problems in a uncomplicated manner. However, in this text, one of the unvisited places in the field is computational complexity. In practice, calculating values is not an easy problem.

The other featured aspect skipped is the variety of forms combinatorial games might take, because the text only deals with partisan games played on normal conditions. The remaining contains a brief description of these two skipped problems and points toward texts on th respective subjects. This chapter, that finalizes the text, also addresses a few of the next steps studying or researching the field and ends with a personal reflection on the learning, implementing and writing experiences involved in this work.


\subsection*{Undeveloped Pieces}

\subsubsection*{Complexity}

The text repeatedly made the case that most concepts are simple in Combinatorial Game Theory. It said, for example, that the method used to find the number a game is equal to is straight-forward. However this hides an important factor.

Either calculating the number or building a thermograph requires, first, traversing the game tree. Traversing this tree and calculating the values is simple enough, but it is slow. And it is slow because this tree quickly becomes large. For example, consider the following game of Domineering.

\begin{center}
\begin{tikzpicture}
\draw[step=0.5cm,color=gray] (-1,-1) grid (1,1);
\newcounter{mycount}
\setcounter{mycount}{`A}
\foreach \y in {+0.75,+0.25,-0.25,-0.75}
\foreach \x in {-0.75,-0.25,0.25,0.75}
\node at (\x,\y) {\addtocounter{mycount}{1}};
\end{tikzpicture}
\end{center}

There are, of course, 25 squares, which means that at most 12 moves might be made and in addition, some of these moves might be equivalent, so they do not have to be computed multiple times. It happens that only 604 nodes, or different position, may be reached\cite{1}. In a $6\times 6$ board it jumps to $17,232$ and in $7\times 7$ and $8\times 8$ board it goes to $408,260$ and $441,990,070$ nodes. Although these results are from 1998, results on larger boards took many years and only in 2016 a result for $11 \times 11$\cite{2} board was found, but abandoning a naive building of a game tree. The game tree of a $9\times 9$ board has around 25 trillion nodes.

If not clear at first it should be now that there is no known form of calculating values for games in polynomial time is respect to their sizes. The question, then, is in what class of problems it fits. It happens that many of the games are known to be PSPACE-complete and some are known to be EXPTIME.

There is of course much more to be said an it is another very interesting topic. A good starting point is \textit{Playing Games with Algorithms: Algorithmic Combinatorial Game Theory}\cite{3}. The thesis \textit{Games, Puzzles, and Computation} is more extensive, covering different classes of games, but may be more complex. Regardless of the suggested texts, there is a large amount of great books on complexity theory.

\subsubsection*{Other games}

By looking at the definition of combinatorial games given in the beginning of section 2, it is easy to come up with games that do not fit well the class of games studied, the short games. One clear example is the game of chess, because there are ties and there are drawn positions.

However, games like chess are not the only class of games that behave differently. Consider G-Hackenbush, a variation of RB-Hackenbush, in which both blue and red edges are replaced by green ones. Green edges may be removed by either player, meaning that both players have the same available moves regardless of the position. The class of games that follow this property is called Impartial Games.

Impartial games were the first games studied, previously to the consolidation of the modern form of the field, by Conway. They have an interesting property of being reducible to an instance of a game of Nim. Typically they are simpler to study but that is not always true. Partisan games, in which players have different sets of available moves may contain components that are impartial, so most of the theory in this text applies to impartial games.

Games like chess, on the other hand, are not so simple. Since there are more possible outcomes than short games analyzing them requires additional elements. Such games benefit from knowing short game's theory but do require further concepts. A good place to start is the corresponding chapter in the book \textit{Combinatorial Game Theory}.


\subsection*{Post-Match}

The choice of Combinatorial Game Theory as subject was an extremely fortunate decision. This field triggered the feeling of uncovering something beautiful hidden in plain sight. I believe the book \textit{Winning Ways} played a role in that due to the presentation I became found of, but I do consider there is inherent beauty in this field of study.

I believe the simplicity of everything about it is something to awe. This is the most important reason why this thesis ended up being about it and not some other subject. The first step in this directions was a great interest in the games of Chess, Shogi and Go. The first idea I had was to develop heuristics to find shortest paths in the game of chess, and possibly extend that to other games.

During that period I started reading articles and scrolling through books and that quickly led me into the field of Combinatorial Game Theory. The book \textit{Winning Ways} is one of the canonical references and I luckily started with that. This choice led me to a year of great learning experiences, in many areas, and personal maturing.

Learning and writing about the topics of this text was the most important activity to conclude my Bachelor's Thesis, but there are other subjects I studied in the process that went unmentioned in the text. I am grateful for those too, as I got to develop my understanding of real analysis and complexity theory further than I would normally, for example. This year we faced a pandemic due to the corona virus widespread. I believe that enjoying the topic as much definitely helped me focus, be proactive and productive in times of isolation where that was hard. 

I had only handful of on-campus classes and that required me, and most of the world, to adapt to new circumstances. While it did not effect the content of this text directly, it greatly impacted my mental state. The consistent work it took really helped keeping me on track. Another factor that cannot go unmentioned is the help of my advisor.

José Coelho kept regularly meetings throughout the year with me. That helped a lot keeping me in check and I am extremely thankful of the dozens of hours we spent talking and discussing the most varied subjects. In the times I did not have work to show he was understanding and his willingness to help pushed me further and further. 

I will definitely continue my studies and work in this area. The field puts together elements of mathematics and computer science in a extremely visual way. I does not only allow one to study while having fun but also gives the sensation of discovery when you apply something or study something new.








