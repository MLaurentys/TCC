\textcolor{red}{This Chapter should demonstrate that you have conducted a thorough and critical investigation of relevant sources.
Apart from a presentation of the sources of your data, this chapter allows you to critically discuss the data (whatever these data are, ‘quantitative’ or ‘qualitative’, primary or secondary), which is proof of good research. You can even do good research with poor data but you must demonstrate that you are aware of the data quality and accordingly are careful in your interpretations. Essentially, there are three aspects to consider:
\begin{enumerate}
\item	Reliability, which, for example, could depend on whether they are estimates or more direct evidence;
\item	Representativity, which is about how typical the data are; for example, you may have arguments why the very few cases are typical or you may carry out statistical tests;
\item Validity, which is about the relevance of the data for your case. Strictly speaking, sometimes no valid data are available but one may argue that there are other data which could be used as ‘proxies’.) 
\end{enumerate}
}
\section{Source Material}
Lorem ipsum dolor sit amet, consectetur adipiscing elit. Duis ut ipsum nec orci interdum sollicitudin ut eu nunc. Pellentesque ultricies eros in justo sagittis, eget blandit velit aliquet. Aenean ac lectus nibh. Quisque ac est pellentesque, ullamcorper sem sit amet, pharetra quam. Morbi ullamcorper placerat diam, sed tincidunt odio.