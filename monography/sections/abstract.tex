\newpage

{
\noindent\Large\textbf{Abstract}
}\\


\noindent
Combinatorial games are called hot when their position is active, meaning that both players want to make the next move. Hot games are the ones that typically lead to an interesting play. Before focusing in this class of games, the text will present fundamental ideas for their study. The first three chapters present and explain the concepts involved with the analysis of combinatorial games. In this first half, there is great concern with the correspondence between games and numbers and how games might or might not be numbers.

The fourth chapter formalizes the idea of temperature and provides methods to handle hot games. The following chapter makes use of all content to provide insight into more advanced ideas and discusses basic applications of the definitions. The sixth chapter, then, works the way up to the proof of a most important theorem and provides its result for a game that permeates the text, Domineering. The text finishes by providing some of the possible next steps, areas that were not discussed and pointing to a few references.

\newpage

{
\noindent\Large\textbf{Resumo}
}\\


\noindent
Jogos combinatórios são chamados quentes quando suas posições são ativas, o que significa que ambos os jogadores querem fazer o próximo movimento. Jogos quentes são aqueles que geralmente proporcionam partidas interessantes. Antes de discutir essa classe de jogos, o texto apresentará ideias fundamentais para seu estudo. Os primeiros três capítulos apresentam e explicam os conceitos envolvidos com análise de jogos combinatórios. Nessa primeira metade, há grande foco na correspondência entre jogos e números e como jogos podem ser ou não números.


O quarto capítulo formaliza a ideia de temperatura e contém métodos para lidar com jogos quentes. O próximo capítulo usa de todo o conteúdo a fim de tratar de problemas mais avançados e discutir aplicações das definições. O sexto capítulo capítulo toma os conteúdos até então apresentados para provar um teorema importante e prover esse resultado para um jogo que permeia o texto, o Domineering. O texto acaba elencando possíveis próximos passos e áreas que não foram discutidas, apontando para algumas outras referências.